\documentclass{article}
\usepackage{ctex}
\usepackage{graphicx}
\usepackage{amsmath}
\usepackage{amsthm}
\usepackage{amssymb}
\usepackage{listings}
\usepackage{geometry}
\usepackage{fontspec}
\usepackage{color}
\usepackage{xcolor}
\usepackage{ulem}
\usepackage{indentfirst}
\definecolor{keywordcolor}{rgb}{0.8,0.1,0.5}

\title{数据结构第四次作业}
\author{鲍桉畅 \quad 软件02 \quad 2020012381}
\date{\today}

\geometry{a4paper,scale=0.8}

\lstset{breaklines}%这条命令可以让LaTeX自动将长的代码行换行排版
\lstset{extendedchars=false}%这一条命令可以解决代码跨页时,章节标题,页眉等汉字不显示的问题
\lstset{
    basicstyle          =   \sffamily,          % 基本代码风格
    keywordstyle        =   \color{keywordcolor} \bfseries,          % 关键字风格
    commentstyle        =   \rmfamily\itshape,  % 注释的风格,斜体
    stringstyle         =   \ttfamily,  % 字符串风格
    flexiblecolumns,                % 别问为什么,加上这个
    numbers             =   left,   % 行号的位置在左边
    showspaces          =   false,  % 是否显示空格,显示了有点乱,所以不现实了
    numberstyle         =   \zihao{-5}\ttfamily,    % 行号的样式,小五号,tt等宽字体
    showstringspaces    =   false,
    captionpos          =   t,      % 这段代码的名字所呈现的位置,t指的是top上面
    language = C++,
}

\begin{document}

\maketitle

\renewcommand{\proofname}{\indent\bf Proof}

\newcommand{\C}{\mathbb{C}}
\newcommand{\R}{\mathbb{R}}
\newcommand{\IM}{\mathbb{Im}}
\newcommand{\Ker}{\mathbb{Ker}}
\newcommand{\rd}{\mathrm{d}}
%\setlength{\parindent}{2em}

\newtheorem{definition}{Definition}
\newtheorem{theorem}{Theorem}
\newtheorem{lemma}{Lemma}
\newtheorem{remark}{Remark}
\newtheorem{proposition}{Proposition}
\newtheorem{corollary}{Corollary}
\newtheorem{example}{Example}
\newtheorem{solution}{Solution}

\begin{solution}
需要进行\textbf{7}次关键字比较
\end{solution}

\begin{solution}
结果为\textbf{ABCD\#\#\#1234}
\end{solution}

\begin{solution}
\textbf{O(n)}
\end{solution}


\begin{solution}
$i = \textbf{5}$,$j = \textbf{2}$
\end{solution}

\begin{solution}
得到的next数组为\textbf{$[-1,0,-1,0,-1,3,0,-1,0]$}
\end{solution}

\begin{solution}
该排序算法只可能是\textbf{C}
\end{solution}

\begin{solution}
\textbf{B}
\end{solution}

\begin{solution}
首个非叶子结点的位置$\frac{n}{2}$,选\textbf{$B$}
\end{solution}

\end{document}