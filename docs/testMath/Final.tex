\documentclass{ctexart}
\usepackage{ctex}
\usepackage{graphicx}
\usepackage{amsmath}
\usepackage{amsthm}
\usepackage{amssymb}
\usepackage{listings}
\usepackage{geometry}
\usepackage{fontspec}
\usepackage{color}
\usepackage{xcolor}
\usepackage{ulem}
\usepackage{indentfirst}
\definecolor{keywordcolor}{rgb}{0.8,0.1,0.5}

\title{复变函数期末复习}
\author{adf wer}
\date{\today}

\geometry{a4paper,scale=0.8}

\lstset{breaklines}%这条命令可以让LaTeX自动将长的代码行换行排版
\lstset{extendedchars=false}%这一条命令可以解决代码跨页时,章节标题,页眉等汉字不显示的问题
\lstset{
    language=C++, %用于设置语言为C++
	keywordstyle=\color{keywordcolor} \bfseries,
	identifierstyle=,
	basicstyle=\ttfamily, 
	commentstyle=\color{blue} \textit,
	stringstyle=\ttfamily, 
	showstringspaces=false,
	%frame=shadowbox, %边框
	captionpos=b
}


\begin{document}

\maketitle

\renewcommand{\proofname}{\indent\bf Proof}
\newcommand{\C}{\mathbb{C}}
\newcommand{\R}{\mathbb{R}}
\newcommand{\IM}{\mathbb{Im}}
\newcommand{\Ker}{\mathbb{Ker}}
\newcommand{\rd}{\mathrm{d}}
%\setlength{\parindent}{2em}

\newtheorem{definition}{Definition}
\newtheorem{theorem}{Theorem}
\newtheorem{lemma}{Lemma}

\newtheorem{remark}{Remark}
\newtheorem{proposition}{Proposition}
\newtheorem{corollary}{Corollary}
\newtheorem{example}{Example}
\newtheorem{solution}{Solution}

\section{复数、复平面、复变函数}

\section{解析函数}

为了讨论解析函数,我们首先需要在复变函数中引入可微与可导的概念,并引入区域中解析函数的概念
\begin{definition}
    设函数$w = f(z)$在区域$D$内处处可微,则称$f(z)$在区域$D$内解析
\end{definition}

解析函数的概念是与区域的概念密切联系的,函数在某点处解析,其意义是指 $f(z)$在该点的某个邻域内解析,在闭域$\bar D$上解析,指的是在包含$\bar D$的某区域解析

通过解析的概念我们可以引入奇点
\begin{definition}
    若函数$f(z)$在某点处不解析,但是在$z_0$的任意邻域内总有$f(z)$的解析点,则称$z_0$为$f(z)$的奇点
\end{definition}

如果一个函数是可微的,那么它的实部和虚部应当不是互相独立的,而必须符合一定的条件。
考虑导数的定义式,容易有
\begin{equation*}
    f'(z) = \lim_{\Delta x \to 0 \\ \Delta y \to 0}\frac{\Delta u + i \Delta v}{\Delta x + i \Delta y}
\end{equation*}
我们分别令$\Delta x = 0, \Delta y = 0$,就能得到如下关系
\begin{equation*}
    f'(z) = u'x + i u'y = v'_y - i u'y
\end{equation*}
比较实部和虚部得到一个偏微分方程组,称为Cauchy-Riemann方程

\begin{theorem}
    \textbf{(可微的必要条件)}设函数$f(z)$在区域$D$内有定义,且在$D$内一点处可微,那么必定有
    \begin{enumerate}
        \item 偏导数$u'_x,u'_y,v'_x,v'_y$存在
        \item 满足Cauchy-Riemann方程
    \end{enumerate}
\end{theorem}

该定理不充分的条件的一个例子是$f(z) = \sqrt{|xy|}$

我们给出可微的一个充要条件
\begin{theorem}
    函数$f(z)$在区域内一点可微的充要条件为
    \begin{enumerate}
        \item 二元函数$u(x,y),v(x.y)$在该点处可微
        \item 该点处满足Cauchy-Riemann方程
    \end{enumerate}
\end{theorem}

于是我们也得到了解析的一个充要条件

值得注意的是,Cauchy-Riemann方程也可以化为形式导数的形式
\begin{equation*}
    \frac{\partial }{\partial \overline{z}} f(z) = 0
\end{equation*}

\begin{definition}
    在复平面上解析的函数称为整函数
\end{definition}

\subsection{初等函数}

\subsubsection{初等单值函数}

首先讨论的函数是指数函数,设

我们定义这样的函数为$f(z)= \exp (z)$:
\begin{enumerate}
    \item $f(z)$为整函数
    \item 满足$f'(z) = f(z)$
    \item 当$y = 0$时,$f(z) = e^x$
\end{enumerate}
我们给出了一个构造$e^z = e^x(cos y + i \sin y$,利用偏微分方程的工具可以证明该函数的唯一性。我们将$e^z$定义为$\exp(z)$

\begin{proposition}
    \begin{equation*}
        e^z = e^{z + 2k\pi i}
    \end{equation*}
\end{proposition}
这一点是值得注意的,另外,这个性质还有逆定理,若$e^{z + \omega} = e^z$,那么可以证明$\omega = 2k\pi ij$

有了指数函数之后可以在复平面上引入三角函数
\begin{equation*}
    \sin z = \frac{e^{iz} - e^{-iz}}{2i} \quad \cos z = \frac{e^{iz} + e^{-iz}}{2}
\end{equation*}

\begin{proposition}
    三角函数的零点集没有发生变化
\end{proposition}
\begin{proposition}
    在复数域内不能断言$|\sin z| \leq 1, |\cos z| \leq 1$
\end{proposition}

双曲函数的定义是类似的
\begin{equation*}
    \sinh z = \frac{e^z -  e^{-z}}{2} \quad \cosh z = \frac{e^z + e^{-z}}{2}
\end{equation*}

\subsubsection{初等多值函数}

为了方便对多值函数的讨论,首先给出下述定义
\begin{definition}
    设函数$f(z)$在区域$D$内有定义,且对$D$内不同的两点$z_1,z_2$,都有$f(z_1) \neq f(z_2)$,则称函数$f(z)$在$D$内是单叶的
\end{definition}

显然,单叶满变换就是一一变换,先来讨论根式函数

\begin{definition}
    根式函数$w = \sqrt[n]{z}$为幂函数$z = w^n$的反函数
\end{definition}
首先找出单叶性区域,变换$z = w^n$将射线$\theta = \phi$映射到$\theta = n\phi$,将圆周$\rho = \rho_0$映射到圆周$\rho = \rho_0^n$。特别地,该变换将角形$-\frac{\pi}{n} < \phi < \frac{\pi}{n}$变成了平面除去负实轴的区域。一般地,平面上$n$个这样的角形都有该性质。



\section{复变函数的积分}

\section{幂级数}

\section{Laurent级数与孤立奇点}

\begin{theorem}
    在圆环域$r < |z - a| < R$内解析的函数一棵可以展开称Laurent级数
    \begin{equation*}
        f(z) = \sum_{n = -\infty}^{+\infty} c_n(z - a)^n
    \end{equation*}
    其中
    \begin{equation*}
        c_n = \frac{1}{2\pi i} \int_{\Gamma} \frac{f(\zeta)}{(\zeta - a) ^{n + 1}} \rd \zeta
    \end{equation*}
    $\Gamma$ 为圆周$|\zeta = a| = \rho$,并且展式是唯一的
\end{theorem}

下面举一些经典例子
\begin{example}
    函数$f(z) = \frac{1}{(z - 1)(z - 2)}$
\end{example}

\section{留数}

\end{document}